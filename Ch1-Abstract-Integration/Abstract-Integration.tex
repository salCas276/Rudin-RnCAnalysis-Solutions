\documentclass{article}
\usepackage{amsmath}
\usepackage{amsfonts}

\author{Salvador Castagnino \\ scastagnino@itba.edu.ar}
\date{} 
\title{Chapter 1 - Abstract Integration}

\begin{document}

\maketitle 

\section*{Exercise Solutions}

\section*{Useful Properties}

\begin{theorem}\textbf{Theorem 1.1 (Reverse Fatou's Lemma)}
    Let $\{f_n\}_{n \in \mathbb{N}}$ be a sequences of measurable functions with $f_n : X \rightarrow [0,\infty]$, if there exsits $g \in L^p \left( \mu \right) $ such that $f_n \le g$ for all $n \ge 1$ then 
    \[
        \limsup_{n \to \infty} \int_X f_n\: d\mu \le \int_X \limsup_{n \to \infty} f_n \: d\mu 
    \]
    \textit{Proof.} As $f_n \le g$ for all $n \ge 1$ the functions $h_n = g-f_n$ are both measurable and nonnegative, thus applying Fatou's Lemma to $h_n$ we get,
    \[
        \int_X \liminf_{n \to \infty} \left( g-f_n \right) \: d\mu \le \liminf_{n \to \infty} \int_X g-f_n\: d\mu 
    \]
    rearranging both sides we get,
    \[
        \int_X g\: d\mu - \limsup_{n \to \infty} \int_X f_n\: d\mu \le \int_X g\: d\mu - \limsup_{n \to \infty} \int_X f_n\: d\mu 
    \]
    Finally, as $\int_X g\: d\mu < \infty$ we can cancel on both sides and multiply by $-1$ getting the desired result.
\end{theorem}

\end{document}
