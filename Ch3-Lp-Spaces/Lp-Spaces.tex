\documentclass{article}
\usepackage{amsmath}
\usepackage{amsfonts}

\author{Salvador Castagnino \\ scastagnino@itba.edu.ar}
\date{}
\title{Chapter 3 - $L^p$ Spaces}

\begin{document}

\maketitle

\begin{exercise}\textbf{Exercise 3.}

    Let's suppose that $\phi$ is not convex in $ \left( a,b \right)   $ then there exist $x,y \in \left( a,b \right) $ and $t_0 \in [0,1]$ such that
    \[
       \phi \left( \left( 1-t_0 \right) x + t_0 y \right) > \left( 1-t_0 \right) \phi \left( x \right) + t_0 \phi \left( y \right)  
    \]
    and assume $x < y$. Let's define $g,h:[0,1] \rightarrow \mathbb{R}$ such that,
    \[
       g \left( t \right) = \phi \left( \left( 1-t \right) x + t y \right)
    \]
    \[
       h \left( t \right) = \left( 1-t \right) \phi \left( x \right) + t \phi \left( y \right)  
        
    \]
    and let $f = h - g$. We define the nonempty sets $A = \{t \in [0,t_0\left)\ |\ f(t)=0\}$ and $B = \{t \in \left(t_0, 1]\ |\ f(t)=0\}$ given that $f \left( 0 \right) = f \left( 1 \right) = 0$, and let $t_a = \sup A$ and $t_b = \inf B$, we want to see that $t_a \in A$  and $t_b \in B$. Let $\{a_n\}$ and $\{b_n\}$ be sequences in $A$ and $B$ which converge to $t_a$ and $t_b$ respectively, by continuity of $f$ we can see that
    \[
        0 = \lim_{n \to \infty} f \left( a_n \right) = f \left( \lim_{n \to \infty} a_n \right) = f \left( t_a \right) 
    \]
    With this and the fact that $t_a < t_0 < t_b$ as $f \left( t_0 \right) < 0$ we have that $t_a \in A$ and the proof for $t_b \in B$ is analogous. Now by the definition of $t_a$ and $t_b$ we have that $f \left( t \right) < 0$ for all $t \in \left( t_a,t_b \right)$. Finally we take $x'= \left( 1-t_a \right) x+t_ay$ and $y' = \left( 1-t_b \right) x+t_by$ and by working algebraically over $f \left( \frac{t_a+t_b}{2} \right) < 0$ we arrive at the following inquality
    \[
        \phi \left( \frac{x'+y'}{2} \right) > \frac{\phi \left( x' \right) + \phi \left( y' \right) }{2}
    \]
    which is a clear contradiction, as we wanted. This concludes the proof.

    
\end{exercise}

\bigbreak

\begin{observation}\textbf{Observation}
    This proof shows that to see that convexity holds it sufices to show that given each pair of points $x,y \in \left( a,b \right) $ there exists a $t \in [0,1]$ (not necessarily $\frac{1}{2}$ ) such that $\phi \left( \left( 1-t \right) x + t y \right) \le  \left( 1-t \right) \phi \left( x \right) + t \phi \left( y \right)$. 
\end{observation}

\bigbreak

\begin{exercise}\textbf{Exercise 4.}
    Let $f$ be a complex measurable function on $X$ and $\mu$ a positive measure on $X$, define
    \[
        \phi \left( p \right) = \int_X |f|^p\: d\mu = \|f\|_p^p \quad \left( 0 < p < \infty \right) 
    \]
    and now let $E = \{p: \phi \left( p \right) < \infty\}$ and assume $0 < \|f\|_\infty$. Let's begin by characterizing $E$. Let $0<p<s<\infty$ and $x \in [0, \infty \left)$ then $x^p < x^s$ iff $1 < x$ and $x^s < x^p$ iff $x < 1$. With this in mind if $0<r<p<s<\infty$ then $|f|^p \le max\{|f|^r,|f|^s\} \le |f|^r+|f|^s$ and we get,
    \[
        \int_X |f|^p\: d\mu \le \int_X |f|^r\: d\mu + \int_X |f|^s\: d\mu 
    \]
    Suppose that $s,r \in E$ then $p \in E$, which proves statement (a).

\bigbreak

    As a consequence of (a) we see that $E$ is a connected set and with this in mind, supposing $E$ is nonempty, we can see that $E$ is an interval with endpoints $a= \inf E$ and $b = \sup E$, this interval will be closed or not if $E$ contains them or not, but for sure $E^o= \left( a,b \right) $.

\bigbreak

    To prove that $\log \phi$ is convex in $E^o$ we start by proving that the composition is well defined, in other words, $\phi$ is never zero. If $\phi \left( p \right) = 0$ for some $p \in E^o $ then $|f|=0$ a.e. on $X$ and this implies $\|f\|_\infty = 0$ contradicting our assumption, thus the composition is well defined and we proceed to prove convexity. Take $x,y \in \left( a,b \right) $ and $t \in \left( 0,1 \right) $. By the properties of the logarithm we get,
    \[
        \left( 1-t \right) \log \phi \left( x \right) + t \log \phi \left( t \right) = \log \left( \phi \left( x\right)^{1-t} \phi \left( y \right)^t \right) 
    \]
    given that the logarithm is a nondecreasing function, convexity holds if and only if
    \[
        \phi \left( (1-t)x + ty \right) \le \phi \left( x \right)^{1-t} \phi \left( y \right)^t
    \]
    rewriting $\phi$ on both sides in terms of Lp norms we get the following,
    \[
        \|f^{ \left( 1-t \right) x} f^{ty}\|_1 \le \|f^{ \left( 1-t \right) x}\|_ {\left( 1-t \right)^{-1} } \|f^{ty}\|_{t^{-1}}   
    \]
    Given that $1 \le \left( 1-t \right)^{-1}$ and $1 \le  t^{-1} $ are conjugate exponents, the last inequallity holds by Holder's inequality and this concludes the proof.

\bigbreak

    To prove that $\phi$ is continous in $E$ we start by noticing that $\log \phi$ is convex in $ \left( a,b \right) $ which implies convexity of $\phi$ in $ \left( a,b \right) $ which in turn implies continuity of $\phi$ in $ \left( a,b \right) $, so we only need to prove continuity on a and b, in case they are elements of $E$.

    Let $\{p_n\}$ be a sequence in $ \left( a,b \right) $ which converges to $a$ and let's suppose $a \in E$, we will show that $\phi \left( p_n \right) $ converges to $\phi \left( a \right) $, the proof for $b$ is analogous. By linearity of the integral and properties of the absolute value we get
    \[
        |\phi \left( p_n \right) - \phi \left( a \right) | = | \int_X |f^{p_n}| - |f^a|\: d\mu | \le \int_X ||f^{p_n}|-|f^a||\: d\mu \le \int_X |f^{p_n} - f^a|\: d\mu 
    \]
    Also, by continuity of the exponential we know that $f^{p_n}$ converges pointwise to $f^a$ as $n$ goes to infinity so it sufices to find a real function $g \in L^1 \left( \mu \right) $ such that $|f^{p_n}| \le g$. Let $g=|f^a|+|f^{p_M}|$ with $p_M$ the greatest element of the sequence $\{p_n\}$ which exists given the convergence of the sequence. Then we can see that $a \le p_n \le p_M$ for all $1 \le n$ and as mentioned above we have
    \[
        |f|^{p_n} \le \max \{ |f|^a , |f|^{p_M}\} \le |f|^a + |f|^{p_M} = g
    \]
    Finally by Lebesgue's Dominated Convergence Theorem we get that
    \[
        \int_X |f^{p_n} - f^a|\: d\mu \rightarrow 0   
    \]
    which concludes the proof of (b).

\bigbreak

\begin{observation}\textbf{Observation}

Let $0 < r < p < s < +\infty$ with $p-r=s-p$ which is the same as $2p=r+s$ then using Holder's inequality we get,  
\[
    \|f\|_p^p = \|f^p\|_1 = \|f^{\frac{r}{2}}f^{\frac{s}{2}}\|_1 \le \|f^{\frac{r}{2}}\|_2 \|f^{\frac{s}{2}}\|_2 = \left( \|f\|_r^r \|f\|_s^s   \right)^{\frac{1}{2}}    
\]
thus,
\[
    \|f\|_p^p \le \sqrt{\|f\|_r^r \|f\|_s^s  }   
\]
\end{observation}

\end{exercise}

\bigbreak

\begin{exercise}\textbf{Exercise 5.}
    In order to prove that $ \|f\|_p \le \|f\|_r$ if $0 < p < r \le \infty$ we start by looking at case in which both $p$ and $r$ are finite. For every $x \in \left( 0,\infty \right) $ the function $x^{c}$ is twice differentiable and we have that
    \[
        \left( x^c \right)'' = \left( x^{c-1}c \right) ' = c \left( c-1 \right) x^{c-2}
    \]
    Observe that if $1 \le c$, the expression above is nonnegative for all $x \in \left( 0,\infty \right)$ and thus $x^c$ convex over that interval. With this in mind and the fact that $\mu \left( \Omega \right) = 1$ and $1 \le \frac{r}{p}$ we will use Jensen's Inequality to prove the result. Let $A = \{x \in \Omega | f \left( x \right) \neq 0 \}$ (the integral over it's complement is exactly $0$ and in this way $|f| \left( A \right) \subset \left( 0,\infty \right)$), then  
    \[
        \|f\|_p = \left( \int_A |f|^p\: d\mu  \right)^{\frac{r}{rp}} \le \left( \int_A |f|^{\frac{rp}{p}}\: d\mu  \right)^\frac{1}{r} = \|f\|_r 
    \]
    and this concludes the proof for the finite case.

    Now let $r = \infty$, this case is much simpler. By definition of the essential supremum we have that $|f| \le \|f\|_\infty $ almost everywhere and using the fact that $\mu \left( \Omega \right) = 1$ we get,
    \[
        \|f\|_p = \left( \int_\Omega |f|^p\: d\mu  \right)^\frac{1}{p} \le \left( \int_\Omega \|f\|_\infty^p \: d\mu  \right) ^\frac{1}{p} = \|f\|_\infty 
    \]
    
\end{exercise}

\bigbreak

\begin{exercise}\textbf{Exercise 7.}

    Let $0 < p < s < \infty$, an example for the inclusion $L^s \left(  \mu\right)  \subset L^p \left( \mu \right) $ can easily be constructed using Ex.5.

    Let $X = \left( 1, \infty \right) $ and $m$ the Lebesgue Measure, we see that neither of the inclusions hold. On the one hand, $\frac{1}{x} \notin L^1 \left( 1, \infty \right)$ and $\frac{1}{x} \in L^2 \left( 1, \infty \right) $. On the other hand, $\frac{I_{ \left( 1,2 \right) }}{ x-1} \notin L^1 \left( 1,\infty \right) $ and $\frac{I_{ \left( 1,2 \right) }}{x-1} \in L^{\frac{1}{2}} \left( 1,\infty \right) $ being $I_A$ the indicator function of $A$. We will only prove the first of these last two statements (the second one can be proved in a simlar manner chosing a convenient partition of $ \left( 1,2 \right) $),
    \[
        \int_{1}^{\infty}\frac{I_{ \left( 1,2 \right) }}{x-1} \: dx  = \int_{1}^{2} \frac{1}{x-1}\: dx = \sum_{n=0}^{\infty} \int_{1+\frac{1}{e^{n+1}}}^{1+\frac{1}{e^n}} \frac{1}{x-1}\: dx 
    \]
    given that the antiderivative of $\frac{1}{x-1}$ is $\ln \left( x-1 \right) + C$ the expression above equals
    \[
        \sum_{n=0}^{\infty} \left [\ln \left( \frac{1}{e^n} \right)  - \ln \left( \frac{1}{e^{n+1}} \right) \right] = \sum_{n=0}^{\infty} \ln \left( e \right) = \infty 
    \]

    For the remaining inclusion we take $X = \mathbb{N}$ with the Counting Measure over the power set of $\mathbb{N}$. Observe that all measurable functions in this case will be just sequences of complex numbers so if $f=\{a_n\}$ we have,
    \[
        \|f\|_p^p = \int_X |f|^p\: d\mu = \sum_{n=1}^{\infty} |a_n|^p 
    \]
    Suppose $\{a_n\} \in \ell^p \left( \mathbb{N} \right)$ then the sequence $\{|a_n|^p\}$ converges to 0 and thus there exists a natural number $M$ such that $|a_n|^p < 1$ for all $M < n$. Given that $1 < \frac{s}{p}$ we have that $|a_n|^s = \left( |a_n|^p \right)^\frac{s}{p} < |a_n|^p$ for all $n$ greater than $M$ and with that in mind we get,
    \[
       \sum_{n=1}^{\infty} |a_n|^s = \sum_{n=1}^{M} |a_n|^s + \sum_{n=M+1}^{\infty} |a_n|^s < \sum_{n=1}^{M} |a_n|^s + \sum_{n=M+1}^{\infty} |a_n|^p < \infty  
    \]
    As we wanted, $\{a_n\} \in \ell ^s \left( \mathbb{N} \right) $, which concludes the proof.
\end{exercise}

\bigbreak

\begin{exercise}$\textbf{Exercise 10.}$
Given that $fg\ge1$ and both $f$ and $g$ are positive we have $f \ge \frac{1}{g}$ and using Holder's inequality and the fact that $\mu\left(\Omega\right)=1$,

\[
    1 = \|1\|_1 = \|g^{\frac{1}{2}}g^{-\frac{1}{2}}\|_1 \le \|g^{\frac{1}{2}}\|_2 \|g^{-\frac{1}{2}}\|_2 =  \left( \|g\|_1 \|g^{-1}\|_1 \right) ^{\frac{1}{2}} \le \left(\|g\|_1 \|f\|_1 \right)^{\frac{1}{2}} 
\]
thus,

\[
    1 \le \int_\Omega f\: d\mu \int_\Omega g\: d\mu  
\]

\end{exercise}

\pagebreak

\begin{exercise}\textbf{Exercise 17.}
   We start by looking at the case in which $1 \le  p < \infty$, we want to see that,
   \begin{equation}
       | \alpha - \beta |^p \le 2^{p-1} \left( |\alpha|^p + |\beta|^p \right) 
   \end{equation}
   for every $\alpha$ and $\beta$ complex numbers. Let $X = [0,1]$ and $\mu$ the Lebesgue Measure, we define the complex function $f$ over $[0,1]$ such that
   \[
       f \left( x \right) =  
       \begin{cases}
           2\alpha & \text{if}\  0 \le x < \frac{1}{2} \\
           2\beta & \text{if}\  \frac{1}{2} \le x \le 1
       \end{cases}
   \]
   Rearrangin $ \left( 1 \right) $ we get,
   \begin{equation}
       |\alpha - \beta| \le \left( \frac{1}{2} \left( 2\alpha \right)^p + \frac{1}{2} \left( 2\beta \right) ^p \right) ^{\frac{1}{p}} = \|f\|_p 
   \end{equation}
   Observing that $\mu \left( X \right) = 1$ and $0 < \|f\|_\infty $ by Ex.5 we know that $\|f\|_1 $ \le $\|f\|_p $ and we have
   \begin{equation}
       |\alpha - \beta| \le |\alpha| + |\beta| = \|f\|_1 \le \|f\|_p 
   \end{equation}
   which concludes the proof. 


\end{exercise}


\end{document}
