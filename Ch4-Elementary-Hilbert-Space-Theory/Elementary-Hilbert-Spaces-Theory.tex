\documentclass{article}
\usepackage{amsmath}
\usepackage{amsfonts}

\author{Salvador Castagnino \\ scastagnino@itba.edu.ar}
\date{}
\title{Chapter 4 - Elementary Hilbert Space Theory}

\begin{document}

\maketitle

\section*{Exercise Solutions}

\begin{exercise}\textbf{Exercise 4.}
    We start by assuming that $\{u_n\}_{n \in \mathbb{N}}$ is a countable maximal orthonormal system in $H$, we want to see that $H$ is separable. Observe that it sufices to find a countable set $A$ such that $P \subset \overline{A}$, with $P$ the set of all finite linear combinations of elements in $\{u_n\}_{n \in \mathbb{N}}$. Define the set $A$ to be,
    \[
        A = \Bigg\{ \sum_{n=1}^{N} \left( q_n + i p_n \right)  u_n : q_n,p_n \in \mathbb{Q},\ N \ge 1\Bigg\} 
    \]
    which is clearly countable. Take $x \in P$, we have $x = \sum_{n=1}^{N} c_n u_n $ for some $c_n$ complex numbers and some $N \ge 1$ . Observe that for every $1 \le n \le N$ there exist $\{q_{nk}\}_{k \in \mathbb{N}}$ and $\{p_{nk}\}_{k \in \mathbb{N}}$ sequences of rational numbers such that $ \left( q_{nk} + i p_{nk} \right) \rightarrow c_n$ as $k$ goes to infinity. Then let $\epsilon > 0$ we can ask for a $k$ large enough such that $|\left( q_{nk} + i p_{nk} \right) - c_n| < \frac{\epsilon}{N}$ for all $1 \le n \le N$, we in turn have
    \[
        \|\sum_{n=1}^{N} \left( q_{nk} + i p_{nk} \right)u_n - \sum_{n=1}^{N} c_n u_n \| \le \sum_{n=1}^{N} |\left( q_{nk} + i p_{nk} \right) - c_n| < \sum_{n=1}^{N} \frac{\epsilon}{N} < \epsilon 
    \]
    Given that every element in $P$ can be aproximated by elements in $A$ we have that, $P \subset \overline{A}$, which concludes the proof.

\bigbreak

Now suppose that $\{x_n\}_{n \in \mathbb{N}}$ is countable and dense in $H$, we are going to build a countable maximal orthonormal system. Let $A$ be the set such that
\[
   \begin{cases}
      x_1 \in A \\
      x_{n+1} \in A \text{ iff } x_{n+1} \notin [ x_1, ..., x_n ]
   \end{cases} 
\]
where $[x_1, ..., x_n]$ denotes the span of $\{x_1, ..., x_n\}$, let's see that $A$ is linearly independent. Suppose that for sume $v_1, ..., v_{m+1}$ in $A$ and $\alpha_1, ..., \alpha_m$ nonzero complex numbers we have that
\[
    \sum_{j=1}^{m} \alpha_j v_j = v_{m+1}
\]
Every $v_j$ can be expressed as $x_{n_j}$, so let $v_k$ be such that $n_k$ is the greates index between all $n_j$, we then have
\[
    v_k = \frac{v_{m+1}}{\alpha_k} - \sum_{1 \le n \neq k \le m} \frac{\alpha_n v_n}{\alpha_k}
\]
which clashes with the construction of A, thus A is linearly independent. By \textbf{Ex4.3} we can build from A a countable orthnormal set $\{u_n\}$ such that $[v_1, ..., v_n] = [u_1, ..., u_n]$ for all $n \ge 1$. Given that every $x_n$ can be expressed as a linear combination of elements in $A$, one can see that every $x_n$ will be in $[u_1, ..., u_m]$ for some $m$. Then we can see that every $x_n$ is an element of $\overline{P}$, with P the set of finite linear combinations of elements in $\{u_n\}$, which in turn imples that $P$ is dense in $H$ and concludes the proof. 
\end{exercise}

\bigbreak

\begin{exercise}\textbf{Exercise 6.}
    We start by supposing that $\sum_{n=1}^{\infty} \delta_n^2 < \infty$, let's prove that $S$ is compact. In ordert to do this, let $\{x_n\}_{n \in \mathbb{N}}$ be a sequence in $S$, we will construct a convergent subsequence.
\begin{itemize}
    \item Let $y_{n1} = \hat{x}_n \left( 1 \right)$, notation as in the book. Given that the closed disk $\overline{B} \left( 0,\delta_1 \right)$ is compact and $y_{n1} \in \overline{B} \left( 0, \delta_1 \right) $, we can find $y_{n_k1}$ a convergent subsequence of $y_{n1}$ which converges to a $y_1$ in that interval. We replace our original sequence $x_n$ for the new subsequence $x_{n_k}$, define $z_1 = x_{n_1}$  and proceed be repeating what we just did but for the second terms (we also rename the new sequence $x_{n_k}$ as $x_n$ for the sake of simplicity).
    \item Notation as above we can find a convergent subsequence $y_{n_k2}$ of the sequence of second terms which converges in $\overline{B} \left( 0,\delta_2 \right) $ to a $y_2$ in that interval. We again define $z_2 = x_{n_2}$ and replace $x_n$ by it's subsequence $x_{n_k}$. Observe that the sequence of first terms still converges as it's a subsequence of a convergent sequence.  
\end{itemize}
Repeating this process an arbirary number of times we build a sequence $z_n$ in S and a sequence of numbers $y_n$ in $\overline{B} \left( 0,\delta_n \right) $ for each $n$. Given that $0 \le |y_n| \le \delta_n$ for all $n \ge 1$ we have that $\sum_{n=1}^{\infty} |y_n|^2 < \infty $ and thus there exists a $y \in S$ such that $\hat{y} \left( n \right) = y_n$ for all $n \ge 1$. By the definition of the sequences $z_n$ and $y_n$, given $\epsilon > 0$ we can find $N, M \in \mathbb{N}$ large enough such that,
\begin{equation}
    \sum_{n=N}^{\infty} |\hat{z}_m \left( n \right)  - y_n|^2 \le \sum_{n=N}^{\infty} 4\delta_n^2   < \frac{\epsilon}{2}
\end{equation}
\begin{equation}
    |\hat{z}_m \left( n \right)  - y_n|^2 < \frac{\epsilon}{2^{n+1}} \quad \forall\ 1 \le n < N
\end{equation}
for all $m \ge M$. With this in mind we can see that $\sum_{n=1}^{\infty} |\hat{z}_m \left( n \right) - y_n|^2 =  \|z_m-y\|<\epsilon$ for all $m \ge M$ which shows that $z_n$ converges to $y$ in $S$ and concludes the proof. 

\bigbreak

Now suppose that $S$ is compact and that $\sum_{n=1}^{\infty} \delta_n^2 = \infty $, we will get to a contradiction. Define $x_k = \sum_{n=1}^{\infty} c_{kn}u_n $ with
\[
    c_{kn} = 
    \begin{cases}
        \delta_n & \text{if } n \le k \\
        0 & \text{else}
    \end{cases}
\]
Clearly $x_k \in S$ for all $k \ge 1$  and $\|x_k\| \rightarrow \sum_{n=1}^{\infty} \delta_n^2 = \infty$ but given that $\|x_k\| \le \|x_k-x\| + \|x\|$ for every $x \in S$ the sequence $x_n$ cannot converge which implies that $S$ is no compact, contradicting our assumption and concluding the proof.  
\end{exercise}

\section*{Useful Properties}

\end{document}
